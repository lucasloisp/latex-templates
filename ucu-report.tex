\documentclass[12pt]{report}
%%%%%%%%%%%%%%%%%%%%%%%%%%%%%%%%%%%%%%%%%%%%%%%%%%%%%%%%%%%%%%%%%%%%%%%%%%%%%%%
%% Document page set-up
\usepackage[a4paper,width=150mm,top=30mm,bottom=30mm]{geometry}
%% Language & Encoding
\usepackage[english]{babel}
%% Prelude
\usepackage[header,prose,code,emoji]{prelude}
%% Bibliography
\addbibresource{bibliografia.bib}
%%%%%%%%%%%%%%%%%%%%%%%%%%%%%%%%%%%%%%%%%%%%%%%%%%%%%%%%%%%%%%%%%%%%%%%%%%%%%%%
\title{Title for the Document}
\subtitle{A subtitle for the document}
\course{Name of the Course}
\institute{Universidad Católica del Uruguay}
\institutelogo{img/ucu}
\author{
	Lucas Lois \and
	Jane Doe
}
\supervisor{John Smith}
\date{\today}
%%%%%%%%%%%%%%%%%%%%%%%%%%%%%%%%%%%%%%%%%%%%%%%%%%%%%%%%%%%%%%%%%%%%%%%%%%%%%%%
\begin{document}
\makeatletter
\begin{titlepage}
    \begin{center}
        \includegraphics[width=0.7\textwidth]{\@institutelogo}

	\vspace*{3cm}

	{\Huge \textbf{\thetitle} \par}

	{\Large \@subtitle \par}

	\vspace*{1.0cm}

	{\LARGE \@course \par}

	\vspace*{3.5cm}

	{\LARGE \theauthor \par}

	\vspace*{1.0cm}

	{\LARGE \@supervisor\\
	{\large Advisor \par} \par}

        \vfill

	{\Large \@institute}

	{\Large \thedate \par}

    \end{center}
\end{titlepage}
\makeatother


\tableofcontents
%%%%%%%%%%%%%%%%%%%%%%%%%%%%%%%%%%%%%%%%%%%%%%%%%%%%%%%%%%%
\chapter{Introduction}

\blindtext \parencite{coulouris}

\section{Emoji Support 🙌}

Hello, 🌎. 🚀🌝.

\section{Syntax Highlighting with Pygments}

This allows the description of types like \mintinline{c}{int} or function calls
like \mintinline{python}{print(x**2)}.
We also allow code blocks, as follows, or even floating listings
(\ref{l:python-hello-world}).

\begin{minted}{c}
int main() {
    printf("hello, world");
    return 0;
}
\end{minted}

\blindtext

\begin{listing}
\begin{minted}{python}
def main():
    print("hello, world")
    return 0
\end{minted}
\caption{We can have code as floats, Python in this case.}
\label{l:python-hello-world}
\end{listing}


%%%%%%%%%%%%%%%%%%%%%%%%%%%%%%%%%%%%%%%%%%%%%%%%%%%%%%%%%%%
\pagebreak
\printbibliography{}
\end{document}

Para poder compartir archivos de forma segura dentro de cada grupo diseñamos
el esquema que sigue (\autoref{fig:cifrado})\label{secc:esquemacifrado}.
\begin{enumerate}
	\item Cada grupo tiene asociado a si una clave \(K_G\), con la cual se
		cifran de forma simétrica, con AES-256 CBC, los archivos
		subidos a ese grupo (\autoref{secc:aes}).
		Ver \code{guardarArchivo(...)} y \code{descargarArchivo(...)}
		en \code{com.""seguridad.""server/""autorizacion.kt}.

	\item 	Como debe ser compartida por todos los usuarios de ese grupo
		usamos un sistema híbrido\autocite{ppt202} similar al de GPG
		para \emph{broadcast encryption} \autocite{gpgbroadcast}.

		\(K_G\) se cifra de forma asimétrica con RSA usando la clave
		pública del usuario, \(Pu_A\).

	\item Guardamos del lado del servidor la clave privada, \(Pr_A\), con
		AES 256 y un clave derivada de la contraseña del usuario,
		\(Pw_A\).
		Esto para que el usuario pueda acceder a toda su información
		desde cualquier dispositivo (una vez en su cuenta).

		La clave se deriva utilizando la función  de derivación de
		claves Argon2 (parámetros en \autoref{secc:argon}).
\end{enumerate}

Antes de cifrar la clave privada, la conviertimos a un \emph{String}
concatenando su \emph{módulo} con su \emph{exponente privado}.
Esto es necesario para poder reconstruirla una vez que es cifrada y
descifrada, dado que esos dos parámetros definen la clave.~\cite{rsa-prk}
Lo mismo hacemos con la clave pública, para poder almacenarla en el servidor.
Esta está definida por los parámetros \emph{módulo} y \emph{exponente público}.~\cite{rsa-puk}

Los datos del sistema se producen de la siguiente manera:

\begin{itemize}
	\item \(K_G\) es de 256 bits y se genera de forma aleatoria al formar
		el nuevo grupo.
	\item  \(Pu_A\) y  \(Pr_A\) son claves RSA de 2048 bits generadas al
		registrar al usuario \(A\) (\autoref{secc:rsa}).
\end{itemize}

\begin{figure}[H]
    \centering
    \incfig{cifrado}
    \caption{Esquema de claves para el cifrado de archivos.
	    El candado indica que el dato está cifrado, y el color su clave.
	    Simétrico o Asimétrico según el tipo de la clave}\label{fig:cifrado}
\end{figure}


\section{Ciclo de Vida de los archivos}

Detallamos a continuación el paso a paso en lo que respecta a la criptografía 
de archivos en el sistema desde que un usuario sube un archivo hasta que otro
miembro del grupo lo baja.

\begin{enumerate}
	\item El usuario \(A\) elige subir el archivo \(F\) al grupo \(G\).
	\item \(A\) propociona su contraseña, \(Pw_A\).
	\item Se toma de la BD el salt asociado a \(A\), \(S_A\).
	\item \(Argon_2(Pw_A, S_A) = Ar_A\), con los parámetros de \autoref{secc:argon}.
	\item \(Ar_A\) descifra, via AES,  \(Pr_a\) de la tabla \(Usuario\).
	\item \(Pr_A\) descifra, via RSA,  \(K_G\) de la tabla \(Pertenece\).
	\item \(K_G\) cifra, via AES, el archivo \(F\).

	\item[] Un tiempo después...
	
	\item El usuario \(B\) elige bajar el archivo \(F\) del grupo \(G\).
	\item \(B\) propociona su contraseña, \(Pw_B\).
	\item Se toma de la BD el salt asociado a \(B\), \(S_B\).
	\item \(Argon_2(Pw_B, S_B) = Ar_B\), con los parámetros de \autoref{secc:argon}.
	\item \(Ar_B\) descifra, via AES,  \(Pr_B\).
	\item \(Pr_B\) descifra, via RSA,  \(K_G\).
	\item \(K_G\) descifra, via AES, el archivo \(F\).
\end{enumerate}

\begin{figure}[H]
    \centering
    \incfig{proceso}
    \caption{Proceso del descifrado de un archivo, indicando qué clave descifra qué.
             Los colores indican las claves según \autoref{fig:cifrado}}\label{fig:proceso}
\end{figure}

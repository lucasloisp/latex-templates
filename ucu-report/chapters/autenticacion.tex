El esquema de autenticación de usuarios es el esquema tradicional de usuarios
que se presentan ante el sistema con nombre de usuario y contraseña.
Una vez que el usuario se autentica de esta forma obtiene un token de acceso
que usa para su autenticación en accesos subsecuentes.\label{secc:autenticacion}

El registro de usuarios se hace con un par \((U, P)\), donde
\(U\) es un valor único en el sistema, codificado como una cadena de caracteres
alfanuméricos, y \(P\) es una cadena de caracteres alfanuméricos o símbolos.

Siguiendo recomendaciones de seguridad \autocite[13]{nist800-63b} requerimos
que la contraseña \(P\) sea de 8 caracteres como mínimo.

En un futuro se puede agregar un protocolo de autenticación en dos pasos como 
TOTP \autocite{rfc6238}.

En la base de datos se almacena el resultado del hash de las contraseñas, usando
bcrypt (\autoref{secc:bcrypt}).
La autenticación se realiza hasheando otra vez la contraseña de forma idéntica
a como se hasheó al almacenar el resultado en la base de datos, usando el mismo
valor de salt. 

El token se genera en cada instancia de autenticación de forma aleatoria (un
vector de 64 bits, 8 bytes, codificado en Base64)\label{secc:token}.
Para generar el vector utilizamos la clase \code{SecureRandom}, una generador
de números randómicos que es \emph{criptográficamente fuerte}
\autocite{SecureRandom}.  En \autocite{nist800-90a} se especifican tres métodos
de generación (DRBG), que se implementaron para \code{SecureRandom} en
\autocite{jep273}.

Solo permitiremos una sesión activa por usuario, requisito que está explicito
en nuestro modelo de datos, donde solo almacenamos un \emph{token} por usuario.


% TODO: Analizar RBAC
% https://security.stackexchange.com/questions/346/what-is-the-difference-between-rbac-and-dac-acl

Para controlar el acceso a los recursos del sistema se utiliza un modelo de 
autorización basado en roles y grupos, en donde el usuario puede pertenecer a  
distintos grupos, pudiendo también crearlos\label{secc:autorization}. 

Esto entra en el marco de un sistema RBAC \autocite{ppt301} ya que, a 
diferencia del modelo discrecional, definimos roles claros para cada usuario
(con respecto de cada grupo) y son esos roles los que indirectamente
determinan los permisos.

Cada rol tiene una configuración de los permisos que se deben interpretar de
él, que se define en la aplicación. 
Estos permisos se representan mediante los atributos booleanos de la clase Rol, 
tal que si su valor es \code{true} se le concede el permiso y si es 
\code{false} no:

\begin{itemize}
    \item \code{puedeAgregar}: permite al usuario agregar a 
    otros usuarios al grupo;
    \item \code{puedeLeer}: permite al usuario leer los 
    archivos pertenecientes al grupo en el cual tiene asignado este permiso;
    \item \code{puedeEscribir}: permite al usuario subir archivos al grupo en 
    el cual tiene asignado este permiso.
\end{itemize}

En la clase Rol se definen los siguentes roles con sus respectivos permisos 
en el formato \code{Rol(puedeAgregar, puedeLeer, puedeEscribir)}: 

\begin{itemize}
    \item \code{ADMINISTRADOR(true, true, true)}
    \item \code{MIEMBRO(false, true, true)}
    \item \code{NINGUNO(false, false, false)}
\end{itemize}

Para más detalle refiérase a \code{com.seguridad.server.Rol}.

Tenemos una cierta forma de herencia ya que, según el modelo de la Base de
Datos y cómo se determina el rol de cada usuario es imposible tener el rol
\code{ADMINISTRADOR} sin tener el rol de \code{MIEMBRO}, es decir que el rol de
administrador hereda del rol de miembro (ya que tiene los mismos permisos y
algunos más).

Un usuario es \emph{miembro} si hay una entrada del par \((Usuario, Grupo)\)
en la tabla \(Pertenece\).
Un usuario es administrador si la entrada del grupo en  \(Grupo\) le indica
como tal.
Nótese que en  \(Grupo\) el par  \((nombre, administrador)\) es clave foránea
del par \((nombreGrupo, nombreUsuario)\) en la tabla \(Pertenece\), i.e.\ no se
puede ser administrador si no se \emph{pertenence} al grupo.
El rol de \code{NINGUNO} se tiene por defecto para cualquier grupo previa la
asignación de un rol mayor.

El sistema desarrollado será una aplicación Cliente/Servidor. 
Los usuarios podrán registrarse y autenticarse en el sistema mediante
el uso de un \emph{nombre de usuario} único y una \emph{contraseña}
(\autoref{secc:autenticacion}).

Una vez ingresado, el usuario podrá ver todos los \emph{grupos} a los
que pertenece, y leer los archivos compartidos en ellos. Un \emph{grupo} 
consiste de uno o más usuarios que tienen acceso a un mismo conjunto de archivos.

Cualquier ``miembro'' de un grupo puede subir archivos a él y leer los archivos
del mismo, pero solamente el administrador (único por grupo) puede agregar
nuevos usuarios como miembros. Para esto debe conocer el nombre de usuario
de la persona que desea agregar (\autoref{secc:autorization}).

La aplicación cliente, es de consola, y permite registrarse, iniciar una sesión, 
crear y ver grupos y bajar y subir archivos.
Para esta última opción se presenta al usuario un cuadro de diálogo para que
indique en qué carpeta guardar un archivo a descargar o qué archivo subir.

Es posible que se solicite al usuario que repita su contraseña para realizar
algunas operaciones críticas (como bajar un archivo, o invitar miembros a un
grupo).

La aplicación servidor, que para los efectos de la defensa ejecutará en el
mismo computador, se presta para ejecutar en un servidor conectado Internet, y
se comunica con el cliente utilizando gRPC (Un protocolo basado en HTTP/2,
alternativo a REST con HTTP+JSON).

